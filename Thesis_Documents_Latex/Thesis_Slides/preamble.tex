\RequirePackage{atbegshi}
\documentclass[aspectratio=169,10pt,xcolor={usenames,dvipsnames,svgnames},hyperref={colorlinks,citecolor=DeepPink4,linkcolor=DarkRed,urlcolor=DarkBlue}]{beamer}



% arara: pdflatex
% !arara: animate: {delay: 80}
% !arara: indent: {overwrite: yes, localSettings: yes}
%\documentclass[mathserif]{beamer}
%\documentclass[handout,mathserif]{beamer}
\usepackage{pgfplots}
\usetikzlibrary{positioning}
\usetikzlibrary{fit}
\usetikzlibrary{backgrounds}
\usetikzlibrary{calc}
\usetikzlibrary{shapes}
\usetikzlibrary{mindmap}
\usetikzlibrary{decorations.text}
\pgfplotsset{compat=1.7}

%\usetheme{Boadilla}
%\usecolortheme{seagull}

% tikzmark command, for shading over items
%\newcommand{\tikzmark}[1]{\tikz[overlay,remember picture] \node (#1) {};}

% standard enumeration
\setbeamertemplate{enumerate items}{(\arabic{enumi})}

% default itemize
\setbeamertemplate{itemize items}[circle]

% transparency
\setbeamercovered{transparent=15}

% for resuming lists across frames
\newcounter{savedenum}
\newcommand*{\saveenum}{\setcounter{savedenum}{\theenumi}}
\newcommand*{\resume}{\setcounter{enumi}{\thesavedenum}}

% title
\tikzset{
   invisible/.style={opacity=0},
   visible on/.style={alt=#1{}{invisible}},
   alt/.code args={<#1>#2#3}{%
      \alt<#1>{\pgfkeysalso{#2}}{\pgfkeysalso{#3}} % \pgfkeysalso doesn't change the path
   },
}

\usepackage{etoolbox}% http://ctan.org/pkg/etoolbox
\makeatletter
\patchcmd{\@listI}{\itemsep3\p@}{\itemsep 0.5\baselineskip}{}{}
\patchcmd{\@listI}{\topsep3\p@}{\topsep 0.5\baselineskip}{}{}
\patchcmd{\@listii}{\itemsep\parsep}{\itemsep 0.5\baselineskip}{}{}
\patchcmd{\@listii}{\topsep2\p@}{\topsep 0.5\baselineskip}{}{}
\patchcmd{\@listiii}{\itemsep\parsep}{\itemsep 0.5\baselineskip}{}{}
\patchcmd{\@listiii}{\topsep2\p@}{\topsep 0.5\baselineskip}{}{}
\makeatother

%% \setlist[1]{\labelindent=\parindent} % < Usually a good idea
%% \setlist[itemize]{leftmargin=*}
%% \setlist[itemize,1]{label=$\triangleleft$}
%% \setlist[enumerate]{labelsep=*, leftmargin=1.5pc}
%% \setlist[enumerate,1]{label=\arabic*., ref=\arabic*}
%% \setlist[enumerate,2]{label=\emph{\alph*}),
%% ref=\theenumi.\emph{\alph*}}
%% \setlist[enumerate,3]{label=\roman*), ref=\theenumii.\roman*}
%% \setlist[description]{font=\sffamily\bfseries}

\setbeamertemplate{itemize items}[ball]
\setbeamertemplate{enumerate items}[ball]

%%\addtolength{\topsep}{0.5\baselineskip}
%%\addtolength{\itemsep}{0.5\baselineskip}

%% \let\oldframe\frame
%% \renewcommand{\frame}{%
%% \oldframe
%% \let\olditemize\itemize
%% \renewcommand\itemize{\olditemize\addtolength{\itemsep}{0.5\baselineskip}\addtolength{\topsep}{0.5\baselineskip}}%
%% }

%
\setbeamertemplate{frametitle}{
    \vspace{\baselineskip}
    \insertframetitle
}



\setbeamertemplate{footline}{
    \vspace{\baselineskip}
}
\setbeamercovered{invisible}

%\usetheme{Pittsburgh}
%\usecolortheme{seahorse}
%\usefonttheme{professionalfonts}

%\setbeamertemplate{background canvas}{\includegraphics [height=\paperheight,width=\paperwidth]{Vintage.png}}
\setbeamertemplate{background canvas}{\includegraphics [height=\paperheight,width=\paperwidth]{background1.jpg}}
%\setbeamertemplate{background canvas}{\includegraphics [height=\paperheight,width=\paperwidth]{Background.png}}
%\setbeamertemplate{background canvas}{\includegraphics [height=\paperheight,width=\paperwidth]{tan-back.png}}

%\setbeamersize{text margin left=4ex,text margin right=4ex}



%\useoutertheme{infolines} 
\newlength{\rcolwidth}
\usepackage{array}
\setlength\extrarowheight{2pt}


\setbeamertemplate{navigation symbols}{}


\mode<handout>
{
  \usepackage{pgf}
  \usepackage{pgfpages}
      
  \pgfpagesdeclarelayout{rotated2up}
  {
    \edef\pgfpageoptionborder{0pt}
  }
  {
    \pgfpagesphysicalpageoptions
    {%
      logical pages=2,%
      physical height=\paperwidth,%
      physical width=\paperheight,%
    }
    \pgfpageslogicalpageoptions{1}
    {%
      resized width=0.75\pgfphysicalwidth,%
      resized height=0.75\pgfphysicalheight,%
      center=\pgfpoint{.5\pgfphysicalwidth}{.75\pgfphysicalheight},%
      border code=\pgfusepath{stroke},
      %rotation=90
    }%
    \pgfpageslogicalpageoptions{2}
    {%
      resized width=.75\pgfphysicalwidth,%
      resized height=.75\pgfphysicalheight,%
      center=\pgfpoint{.5\pgfphysicalwidth}{.25\pgfphysicalheight},
      border code=\pgfusepath{stroke},
      %rotation=90
    }%
  }

  \pgfpagesuselayout{rotated2up}%[portrait]
  \nofiles
}

\usepackage{pgffor}
%\usepackage{algorithmic}
%\usepackage{xlop}
%\usepackage{amsmath}
\usepackage{listings}
%\usepackage{../texmacros/arm}
%\usepackage{../texmacros/fixedpoint}
\usepackage{tikz}
\usetikzlibrary{%
  arrows,
  calc,
  decorations,
  decorations.pathreplacing,
  decorations.pathmorphing,
  shapes
}


\usepackage{multirow}
\usepackage{bigdelim}

\usepackage{xcolor}
\usepackage{multicol}
\usepackage{xspace}

%\usetheme{default}
%\usetheme{Madrid} %nice


%\usetheme[height=1.1\baselineskip]{Rochester}



%\usetheme{default}
%\usecolortheme{rose}

%\usetheme{progressbar}



%\usefont{serif}


%\usetheme{Boadilla}
%\usetheme{Frankfurt}
%\setbeamertemplate{navigation symbols}{}
%\usecolortheme{rose}

%\usecolortheme{seahorse}
%\usecolortheme{orchid}
\usecolortheme{rose}
%\usecolortheme{seagull}

%\usefonttheme[onlymath]{serif}
\usefonttheme{serif}


%\useoutertheme{infolines} 


%\input{ieeefloats.tex}

\newenvironment{Ventry}[1]%
{\begin{list}{}{\renewcommand{\makelabel}[1]{\bf{##1:}\hfil}\settowidth{\labelwidth}{\bf{#1:}}\setlength{\leftmargin}{\labelwidth+\labelsep}}}{\end{list}}

\newsavebox{\tmpbox}
\newsavebox{\tmpboxb}

%\newcommand{\tikzmark}[1]{\tikz[overlay,remember picture] \node (#1) {};}

%% \newcommand*{\BraceAmplitude}{0.5em}%  Can be tweaked if
%% \newcommand*{\VerticalOffset}{1.2ex}%  necessary.
%% \newcommand*{\HorizontalOffset}{0.3em}%  necessary.
%% \newcommand*{\InsertUnderBrace}[4][]{%
%%     \begin{tikzpicture}[overlay,remember picture]
%% \draw [decoration={brace,amplitude=\BraceAmplitude},decorate, thick,draw=black,text=black,#1]
%%         ($(#3)+(\HorizontalOffset,-\VerticalOffset)$) -- 
%%         ($(#2)+(-\HorizontalOffset,-\VerticalOffset)$)
%%         node [below=\VerticalOffset, midway] {#4};
%%     \end{tikzpicture}%
%% }%

%% \tikzstyle{line} = [draw, very thick, color=black!80, -latex']

\newlength{\boxwidth}
\settowidth{\boxwidth}{U(7,9)}
\addtolength{\boxwidth}{1em}

\newcommand{\lstvdots}{\raisebox{-0.25\baselineskip}{\mbox{$\smash{\vdots\strut}$}}}

% used to make sure that lstlisting aligns every character in a column
\newlength{\normalbasewidth}
\settowidth{\normalbasewidth}{\ttfamily m}
\newlength{\smallbasewidth}
\settowidth{\smallbasewidth}{\ttfamily\small m}
\newlength{\footnotebasewidth}
\settowidth{\footnotebasewidth}{\ttfamily\footnotesize M\rule{0.5pt}{0.5pt}}
\newlength{\scriptbasewidth}
\settowidth{\scriptbasewidth}{\ttfamily\scriptsize M\rule{0.5pt}{0.5pt}}
\newlength{\tinybasewidth}
\settowidth{\tinybasewidth}{\tiny\ttfamily M\rule{0.5pt}{0.5pt}}

\makeatletter
\lstdefinestyle{normalstyle}{
  basewidth=\normalbasewidth,
  basicstyle=\ttfamily\lst@ifdisplaystyle\ttfamily\normalsize\fi
}
\makeatother

\makeatletter
\lstdefinestyle{smallstyle}{
  basewidth=\smallbasewidth,
  basicstyle=\ttfamily\lst@ifdisplaystyle\ttfamily\small\fi
}
\makeatother


\makeatletter
\lstdefinestyle{mystyle}{
  basewidth=\footnotebasewidth,
  basicstyle=\ttfamily\lst@ifdisplaystyle\ttfamily\footnotesize\fi
}
\makeatother

\makeatletter
\lstdefinestyle{footnotestyle}{
  basewidth=\footnotebasewidth,
  basicstyle=\ttfamily\lst@ifdisplaystyle\ttfamily\footnotesize\fi
}
\makeatother

\makeatletter
\lstdefinestyle{scriptstyle}{
  basewidth=\scriptbasewidth,
  basicstyle=\ttfamily\lst@ifdisplaystyle\ttfamily\scriptsize\fi
}
\makeatother

\makeatletter
\lstdefinestyle{tinystyle}{
  basewidth=\tinybasewidth,
  basicstyle=\tiny\ttfamily\lst@ifdisplaystyle\tiny\ttfamily\fi
}
\makeatother

\definecolor{lightbackground}{RGB}{240,240,255}
\definecolor{lightbackground}{RGB}{250,250,255}
\definecolor{lightbackground}{RGB}{255,255,240}


\lstset{language=VHDL,%[arm]Assembler,
  backgroundcolor=\color{white},  % choose the background color; you must add \usepackage{color} or \usepackage{xcolor}
    backgroundcolor=\color{lightgray},  % choose the background color; you must add \usepackage{color} or \usepackage{xcolor}  
    backgroundcolor=\color{lightbackground},  % choose the background color; you must add \usepackage{color} or \usepackage{xcolor}  
  %basicstyle=\ttfamily\footnotesize,
  style=normalstyle,
  breakatwhitespace=false,        % sets if automatic breaks should only happen at whitespace
  %linewidth={0.9\textwidth},
  xleftmargin=8pt,
  xrightmargin=8pt,
  breaklines=false,               % sets automatic line breaking
  captionpos=b,                   % sets the caption-position to bottom
  commentstyle=\color{OliveGreen},   % comment style
  commentstyle=\color{RedViolet},   % comment style
  commentstyle=\color{WildStrawberry},   % comment style
  commentstyle=\color{JungleGreen},   % comment style
  commentstyle=\color{ForestGreen},   % comment style
  commentstyle=\color{Purple},   % comment style
  %escapechar=`,
  %escapeinside={\%*}{*)},         % if you want to add LaTeX within your code
  %extendedchar=true,              % lets you use non-ASCII characters; for 8-bits encodings only, does not work with UTF-8
  frame=single,                   % adds a frame around the code
  keywordstyle=[1]\color{blue},      % keyword style
  keywordstyle=[2]\color{cyan},      % keyword style
  keywordstyle=[3]\color{Bittersweet},
  numbers=left,                   % where to put the line-numbers; possible values are (none, left, right)
  numbersep=5pt,                  % how far the line-numbers are from the code
  numberstyle=\tiny\color{gray},  % the style that is used for the line-numbers
  rulecolor=\color{black},        % if not set, the frame-color may be changed on line-breaks within not-black text (e.g. comments (green here))
  showspaces=false,               % show spaces everywhere adding particular underscores; it overrides 'showstringspaces'
  showstringspaces=false,         % underline spaces within strings only
  showtabs=false,                 % show tabs within strings adding particular underscores
  stepnumber=1,                   % the step between two line-numbers. If it's 1, each line will be numbered
  stringstyle=\color{purple},      % string literal style
  tabsize=8                      % sets default tabsize to 8 spaces
}

\usepackage{fmtcount}

%\usepackage{../texmacros/longdiv}



\usepackage{adjustbox}

%\usepackage[paperwidth=191mm,paperheight=235mm,bindingoffset=20mm,textwidth=131mm,textheight=195mm,top=20mm]{geometry}

\usepackage{caption}
\usepackage[within=section]{newfloat}
\usepackage{proof}
\usepackage{makeidx}
\usepackage{rotating}
%% \usepackage{keyval}    
\usepackage{newcent} % nice, medium width font
%\usepackage{fouriernc} % compatible math font
\usepackage[T1]{fontenc}
\usepackage{amsmath}
\usepackage{amssymb}
\usepackage{bigstrut}
\usepackage{array}
\usepackage{settobox}
\usepackage{mdframed}
\usepackage{needspace}
\usepackage{cprotect}
\usepackage{textcomp}
%\usepackage{subcaption}
\usepackage[countmax]{subfloat}


\usepackage{changepage}
%\usepackage{lipsum}
%\usepackage{tocloft}
\usepackage[usenames,dvipsnames,svgnames]{xcolor}
%\usepackage{../texmacros/scalerel}
%\usepackage{../texmacros/arm}
%\usepackage{../texmacros/longdiv}
%\usepackage{../texmacros/fixedpoint}
\usepackage{framed}
\usepackage{calc}
\usepackage{fancyvrb}
\usepackage{cancel}
%\usepackage{algorithmic}

%\usepackage{algorithmicx}
%\usepackage{algpseudocode}
%\usepackage[plain]{algorithm}

\usepackage{xlop}

%\usepackage{enumitem}


%\usepackage{hhline}
%\usepackage{multirow}
%\usepackage{bigdelim}
%\usepackage{multicol}
\usepackage{longtable}
\usepackage{tabu}
\usepackage{tabularx}
\usepackage{tabulary}
%\usepackage{xspace}
\usepackage{eurosym} 
%\usepackage{tikz}
%\usepackage{environ}
%\usepackage{etoolbox}
%\usepackage{caption}
\usepackage{adjustbox}
%\usepackage{hyperref}
%\usepackage{hypcap}
%\usepackage{etex} 
\usepackage{morewrites}
%
\usepackage{tipa}
%\usepackage{t4phonet}

%\lstset{language=[arm]Assembler}

\title[CENG 342]{CENG 342 -- Digital Systems}
\author{Larry Pyeatt}
\institute{SDSM\&T}
\date{}

\newlength{\mywidth}
\settowidth{\mywidth}{0}
\addtolength{\mywidth}{2.8pt}


\newcommand{\valuelistw}[2]{
\noindent
%\hspace*{-2ex}
\begin{minipage}{0.6\textwidth}
\vspace{0.1\baselineskip}
\begin{enumerate}%[leftmargin=*,label=(\alph),align=left,widest={#1}]
\setlength\itemsep{-0.25\baselineskip}#2
\end{enumerate}\vspace{-3pt}
\end{minipage}}

 
\newcommand{\hexnum}[1]{$\texttt{#1}_{\texttt{16}}$\xspace}
\newcommand{\binnum}[1]{$\texttt{#1}_{\texttt{2}}$\xspace}

%% \usepackage{enumitem}

%% \newenvironment{mydesc}[1]
%%   {\settowidth{\dimen0}{#1}%
%%    \renewcommand{\descriptionlabel}[1]{##1\hfill}%
%%    \begin{description}[leftmargin=\dimexpr\dimen0+\labelsep\relax,labelwidth=\dimen0 ]}
%%   {\end{description}}
  
%% %\setlist[1]{\labelindent=\parindent} % < Usually a good idea
%% \setlist[itemize]{leftmargin=*}
%% \setlist[itemize,1]{label=\textbullet}

\newenvironment{mydesc}[1]
  {\list{}{\renewcommand\makelabel[1]{##1\hfil}\setlength{\itemsep}{1pt}%
     \settowidth\labelwidth{\makelabel{#1}}%
     \setlength\leftmargin{\dimexpr\labelwidth+\labelsep\relax}}}
  {\endlist}


\usepackage{hhline}
\usepackage{adjustbox}


\makeatletter
\def\lst@MSkipToFirst{%
    \global\advance\lst@lineno\@ne
    \ifnum \lst@lineno=\lst@firstline
        \def\lst@next{\lst@LeaveMode \global\lst@newlines\z@
        \lst@OnceAtEOL \global\let\lst@OnceAtEOL\@empty
        \lst@InitLstNumber % Added to work with modified \lsthk@PreInit.
        \lsthk@InitVarsBOL
        \c@lstnumber=\numexpr-1+\lst@lineno % this enforces the displayed line numbers to always be the input line numbers
        \lst@BOLGobble}%
        \expandafter\lst@next
    \fi}
\makeatother

\usepackage{tikz-timing}
\usepackage{graphbox}% http://ctan.org/pkg/adjustbox
% \usepackage[10pt]{rail}
% \input{/home/faculty/lpyeatt/courses/VHDL_book/newrails}



\definecolor{DarkBlue}{rgb}{0.0, 0.0, 0.55}
