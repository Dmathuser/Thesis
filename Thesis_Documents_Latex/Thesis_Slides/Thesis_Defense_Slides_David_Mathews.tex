\RequirePackage{atbegshi}
\documentclass[aspectratio=169,10pt,xcolor={usenames,dvipsnames,svgnames},hyperref={colorlinks,citecolor=DeepPink4,linkcolor=DarkRed,urlcolor=DarkBlue}]{beamer}



% arara: pdflatex
% !arara: animate: {delay: 80}
% !arara: indent: {overwrite: yes, localSettings: yes}
%\documentclass[mathserif]{beamer}
%\documentclass[handout,mathserif]{beamer}
\usepackage{pgfplots}
\usetikzlibrary{positioning}
\usetikzlibrary{fit}
\usetikzlibrary{backgrounds}
\usetikzlibrary{calc}
\usetikzlibrary{shapes}
\usetikzlibrary{mindmap}
\usetikzlibrary{decorations.text}
\pgfplotsset{compat=1.7}

%\usetheme{Boadilla}
%\usecolortheme{seagull}

% tikzmark command, for shading over items
%\newcommand{\tikzmark}[1]{\tikz[overlay,remember picture] \node (#1) {};}

% standard enumeration
\setbeamertemplate{enumerate items}{(\arabic{enumi})}

% default itemize
\setbeamertemplate{itemize items}[circle]

% transparency
\setbeamercovered{transparent=15}

% for resuming lists across frames
\newcounter{savedenum}
\newcommand*{\saveenum}{\setcounter{savedenum}{\theenumi}}
\newcommand*{\resume}{\setcounter{enumi}{\thesavedenum}}

% title
\tikzset{
   invisible/.style={opacity=0},
   visible on/.style={alt=#1{}{invisible}},
   alt/.code args={<#1>#2#3}{%
      \alt<#1>{\pgfkeysalso{#2}}{\pgfkeysalso{#3}} % \pgfkeysalso doesn't change the path
   },
}

\usepackage{etoolbox}% http://ctan.org/pkg/etoolbox
\makeatletter
\patchcmd{\@listI}{\itemsep3\p@}{\itemsep 0.5\baselineskip}{}{}
\patchcmd{\@listI}{\topsep3\p@}{\topsep 0.5\baselineskip}{}{}
\patchcmd{\@listii}{\itemsep\parsep}{\itemsep 0.5\baselineskip}{}{}
\patchcmd{\@listii}{\topsep2\p@}{\topsep 0.5\baselineskip}{}{}
\patchcmd{\@listiii}{\itemsep\parsep}{\itemsep 0.5\baselineskip}{}{}
\patchcmd{\@listiii}{\topsep2\p@}{\topsep 0.5\baselineskip}{}{}
\makeatother

%% \setlist[1]{\labelindent=\parindent} % < Usually a good idea
%% \setlist[itemize]{leftmargin=*}
%% \setlist[itemize,1]{label=$\triangleleft$}
%% \setlist[enumerate]{labelsep=*, leftmargin=1.5pc}
%% \setlist[enumerate,1]{label=\arabic*., ref=\arabic*}
%% \setlist[enumerate,2]{label=\emph{\alph*}),
%% ref=\theenumi.\emph{\alph*}}
%% \setlist[enumerate,3]{label=\roman*), ref=\theenumii.\roman*}
%% \setlist[description]{font=\sffamily\bfseries}

\setbeamertemplate{itemize items}[ball]
\setbeamertemplate{enumerate items}[ball]

%%\addtolength{\topsep}{0.5\baselineskip}
%%\addtolength{\itemsep}{0.5\baselineskip}

%% \let\oldframe\frame
%% \renewcommand{\frame}{%
%% \oldframe
%% \let\olditemize\itemize
%% \renewcommand\itemize{\olditemize\addtolength{\itemsep}{0.5\baselineskip}\addtolength{\topsep}{0.5\baselineskip}}%
%% }

%
\setbeamertemplate{frametitle}{
    \vspace{\baselineskip}
    \insertframetitle
}



\setbeamertemplate{footline}{
    \vspace{\baselineskip}
}
\setbeamercovered{invisible}

%\usetheme{Pittsburgh}
%\usecolortheme{seahorse}
%\usefonttheme{professionalfonts}

%\setbeamertemplate{background canvas}{\includegraphics [height=\paperheight,width=\paperwidth]{Vintage.png}}
\setbeamertemplate{background canvas}{\includegraphics [height=\paperheight,width=\paperwidth]{background1.jpg}}
%\setbeamertemplate{background canvas}{\includegraphics [height=\paperheight,width=\paperwidth]{Background.png}}
%\setbeamertemplate{background canvas}{\includegraphics [height=\paperheight,width=\paperwidth]{tan-back.png}}

%\setbeamersize{text margin left=4ex,text margin right=4ex}



%\useoutertheme{infolines} 
\newlength{\rcolwidth}
\usepackage{array}
\setlength\extrarowheight{2pt}


\setbeamertemplate{navigation symbols}{}


\mode<handout>
{
  \usepackage{pgf}
  \usepackage{pgfpages}
      
  \pgfpagesdeclarelayout{rotated2up}
  {
    \edef\pgfpageoptionborder{0pt}
  }
  {
    \pgfpagesphysicalpageoptions
    {%
      logical pages=2,%
      physical height=\paperwidth,%
      physical width=\paperheight,%
    }
    \pgfpageslogicalpageoptions{1}
    {%
      resized width=0.75\pgfphysicalwidth,%
      resized height=0.75\pgfphysicalheight,%
      center=\pgfpoint{.5\pgfphysicalwidth}{.75\pgfphysicalheight},%
      border code=\pgfusepath{stroke},
      %rotation=90
    }%
    \pgfpageslogicalpageoptions{2}
    {%
      resized width=.75\pgfphysicalwidth,%
      resized height=.75\pgfphysicalheight,%
      center=\pgfpoint{.5\pgfphysicalwidth}{.25\pgfphysicalheight},
      border code=\pgfusepath{stroke},
      %rotation=90
    }%
  }

  \pgfpagesuselayout{rotated2up}%[portrait]
  \nofiles
}

\usepackage{pgffor}
%\usepackage{algorithmic}
%\usepackage{xlop}
%\usepackage{amsmath}
\usepackage{listings}
%\usepackage{../texmacros/arm}
%\usepackage{../texmacros/fixedpoint}
\usepackage{tikz}
\usetikzlibrary{%
  arrows,
  calc,
  decorations,
  decorations.pathreplacing,
  decorations.pathmorphing,
  shapes
}


\usepackage{multirow}
\usepackage{bigdelim}

\usepackage{xcolor}
\usepackage{multicol}
\usepackage{xspace}

%\usetheme{default}
%\usetheme{Madrid} %nice


%\usetheme[height=1.1\baselineskip]{Rochester}



%\usetheme{default}
%\usecolortheme{rose}

%\usetheme{progressbar}



%\usefont{serif}


%\usetheme{Boadilla}
%\usetheme{Frankfurt}
%\setbeamertemplate{navigation symbols}{}
%\usecolortheme{rose}

%\usecolortheme{seahorse}
%\usecolortheme{orchid}
\usecolortheme{rose}
%\usecolortheme{seagull}

%\usefonttheme[onlymath]{serif}
\usefonttheme{serif}


%\useoutertheme{infolines} 


%\input{ieeefloats.tex}

\newenvironment{Ventry}[1]%
{\begin{list}{}{\renewcommand{\makelabel}[1]{\bf{##1:}\hfil}\settowidth{\labelwidth}{\bf{#1:}}\setlength{\leftmargin}{\labelwidth+\labelsep}}}{\end{list}}

\newsavebox{\tmpbox}
\newsavebox{\tmpboxb}

%\newcommand{\tikzmark}[1]{\tikz[overlay,remember picture] \node (#1) {};}

%% \newcommand*{\BraceAmplitude}{0.5em}%  Can be tweaked if
%% \newcommand*{\VerticalOffset}{1.2ex}%  necessary.
%% \newcommand*{\HorizontalOffset}{0.3em}%  necessary.
%% \newcommand*{\InsertUnderBrace}[4][]{%
%%     \begin{tikzpicture}[overlay,remember picture]
%% \draw [decoration={brace,amplitude=\BraceAmplitude},decorate, thick,draw=black,text=black,#1]
%%         ($(#3)+(\HorizontalOffset,-\VerticalOffset)$) -- 
%%         ($(#2)+(-\HorizontalOffset,-\VerticalOffset)$)
%%         node [below=\VerticalOffset, midway] {#4};
%%     \end{tikzpicture}%
%% }%

%% \tikzstyle{line} = [draw, very thick, color=black!80, -latex']

\newlength{\boxwidth}
\settowidth{\boxwidth}{U(7,9)}
\addtolength{\boxwidth}{1em}

\newcommand{\lstvdots}{\raisebox{-0.25\baselineskip}{\mbox{$\smash{\vdots\strut}$}}}

% used to make sure that lstlisting aligns every character in a column
\newlength{\normalbasewidth}
\settowidth{\normalbasewidth}{\ttfamily m}
\newlength{\smallbasewidth}
\settowidth{\smallbasewidth}{\ttfamily\small m}
\newlength{\footnotebasewidth}
\settowidth{\footnotebasewidth}{\ttfamily\footnotesize M\rule{0.5pt}{0.5pt}}
\newlength{\scriptbasewidth}
\settowidth{\scriptbasewidth}{\ttfamily\scriptsize M\rule{0.5pt}{0.5pt}}
\newlength{\tinybasewidth}
\settowidth{\tinybasewidth}{\tiny\ttfamily M\rule{0.5pt}{0.5pt}}

\makeatletter
\lstdefinestyle{normalstyle}{
  basewidth=\normalbasewidth,
  basicstyle=\ttfamily\lst@ifdisplaystyle\ttfamily\normalsize\fi
}
\makeatother

\makeatletter
\lstdefinestyle{smallstyle}{
  basewidth=\smallbasewidth,
  basicstyle=\ttfamily\lst@ifdisplaystyle\ttfamily\small\fi
}
\makeatother


\makeatletter
\lstdefinestyle{mystyle}{
  basewidth=\footnotebasewidth,
  basicstyle=\ttfamily\lst@ifdisplaystyle\ttfamily\footnotesize\fi
}
\makeatother

\makeatletter
\lstdefinestyle{footnotestyle}{
  basewidth=\footnotebasewidth,
  basicstyle=\ttfamily\lst@ifdisplaystyle\ttfamily\footnotesize\fi
}
\makeatother

\makeatletter
\lstdefinestyle{scriptstyle}{
  basewidth=\scriptbasewidth,
  basicstyle=\ttfamily\lst@ifdisplaystyle\ttfamily\scriptsize\fi
}
\makeatother

\makeatletter
\lstdefinestyle{tinystyle}{
  basewidth=\tinybasewidth,
  basicstyle=\tiny\ttfamily\lst@ifdisplaystyle\tiny\ttfamily\fi
}
\makeatother

\definecolor{lightbackground}{RGB}{240,240,255}
\definecolor{lightbackground}{RGB}{250,250,255}
\definecolor{lightbackground}{RGB}{255,255,240}


\lstset{language=VHDL,%[arm]Assembler,
  backgroundcolor=\color{white},  % choose the background color; you must add \usepackage{color} or \usepackage{xcolor}
    backgroundcolor=\color{lightgray},  % choose the background color; you must add \usepackage{color} or \usepackage{xcolor}  
    backgroundcolor=\color{lightbackground},  % choose the background color; you must add \usepackage{color} or \usepackage{xcolor}  
  %basicstyle=\ttfamily\footnotesize,
  style=normalstyle,
  breakatwhitespace=false,        % sets if automatic breaks should only happen at whitespace
  %linewidth={0.9\textwidth},
  xleftmargin=8pt,
  xrightmargin=8pt,
  breaklines=false,               % sets automatic line breaking
  captionpos=b,                   % sets the caption-position to bottom
  commentstyle=\color{OliveGreen},   % comment style
  commentstyle=\color{RedViolet},   % comment style
  commentstyle=\color{WildStrawberry},   % comment style
  commentstyle=\color{JungleGreen},   % comment style
  commentstyle=\color{ForestGreen},   % comment style
  commentstyle=\color{Purple},   % comment style
  %escapechar=`,
  %escapeinside={\%*}{*)},         % if you want to add LaTeX within your code
  %extendedchar=true,              % lets you use non-ASCII characters; for 8-bits encodings only, does not work with UTF-8
  frame=single,                   % adds a frame around the code
  keywordstyle=[1]\color{blue},      % keyword style
  keywordstyle=[2]\color{cyan},      % keyword style
  keywordstyle=[3]\color{Bittersweet},
  numbers=left,                   % where to put the line-numbers; possible values are (none, left, right)
  numbersep=5pt,                  % how far the line-numbers are from the code
  numberstyle=\tiny\color{gray},  % the style that is used for the line-numbers
  rulecolor=\color{black},        % if not set, the frame-color may be changed on line-breaks within not-black text (e.g. comments (green here))
  showspaces=false,               % show spaces everywhere adding particular underscores; it overrides 'showstringspaces'
  showstringspaces=false,         % underline spaces within strings only
  showtabs=false,                 % show tabs within strings adding particular underscores
  stepnumber=1,                   % the step between two line-numbers. If it's 1, each line will be numbered
  stringstyle=\color{purple},      % string literal style
  tabsize=8                      % sets default tabsize to 8 spaces
}

\usepackage{fmtcount}

%\usepackage{../texmacros/longdiv}



\usepackage{adjustbox}

%\usepackage[paperwidth=191mm,paperheight=235mm,bindingoffset=20mm,textwidth=131mm,textheight=195mm,top=20mm]{geometry}

\usepackage{caption}
\usepackage[within=section]{newfloat}
\usepackage{proof}
\usepackage{makeidx}
\usepackage{rotating}
%% \usepackage{keyval}    
\usepackage{newcent} % nice, medium width font
%\usepackage{fouriernc} % compatible math font
\usepackage[T1]{fontenc}
\usepackage{amsmath}
\usepackage{amssymb}
\usepackage{bigstrut}
\usepackage{array}
\usepackage{settobox}
\usepackage{mdframed}
\usepackage{needspace}
\usepackage{cprotect}
\usepackage{textcomp}
%\usepackage{subcaption}
\usepackage[countmax]{subfloat}


\usepackage{changepage}
%\usepackage{lipsum}
%\usepackage{tocloft}
\usepackage[usenames,dvipsnames,svgnames]{xcolor}
%\usepackage{../texmacros/scalerel}
%\usepackage{../texmacros/arm}
%\usepackage{../texmacros/longdiv}
%\usepackage{../texmacros/fixedpoint}
\usepackage{framed}
\usepackage{calc}
\usepackage{fancyvrb}
\usepackage{cancel}
%\usepackage{algorithmic}

%\usepackage{algorithmicx}
%\usepackage{algpseudocode}
%\usepackage[plain]{algorithm}

\usepackage{xlop}

%\usepackage{enumitem}


%\usepackage{hhline}
%\usepackage{multirow}
%\usepackage{bigdelim}
%\usepackage{multicol}
\usepackage{longtable}
\usepackage{tabu}
\usepackage{tabularx}
\usepackage{tabulary}
%\usepackage{xspace}
\usepackage{eurosym} 
%\usepackage{tikz}
%\usepackage{environ}
%\usepackage{etoolbox}
%\usepackage{caption}
\usepackage{adjustbox}
%\usepackage{hyperref}
%\usepackage{hypcap}
%\usepackage{etex} 
\usepackage{morewrites}
%
\usepackage{tipa}
%\usepackage{t4phonet}

%\lstset{language=[arm]Assembler}

\title[CENG 342]{CENG 342 -- Digital Systems}
\author{Larry Pyeatt}
\institute{SDSM\&T}
\date{}

\newlength{\mywidth}
\settowidth{\mywidth}{0}
\addtolength{\mywidth}{2.8pt}


\newcommand{\valuelistw}[2]{
\noindent
%\hspace*{-2ex}
\begin{minipage}{0.6\textwidth}
\vspace{0.1\baselineskip}
\begin{enumerate}%[leftmargin=*,label=(\alph),align=left,widest={#1}]
\setlength\itemsep{-0.25\baselineskip}#2
\end{enumerate}\vspace{-3pt}
\end{minipage}}

 
\newcommand{\hexnum}[1]{$\texttt{#1}_{\texttt{16}}$\xspace}
\newcommand{\binnum}[1]{$\texttt{#1}_{\texttt{2}}$\xspace}

%% \usepackage{enumitem}

%% \newenvironment{mydesc}[1]
%%   {\settowidth{\dimen0}{#1}%
%%    \renewcommand{\descriptionlabel}[1]{##1\hfill}%
%%    \begin{description}[leftmargin=\dimexpr\dimen0+\labelsep\relax,labelwidth=\dimen0 ]}
%%   {\end{description}}
  
%% %\setlist[1]{\labelindent=\parindent} % < Usually a good idea
%% \setlist[itemize]{leftmargin=*}
%% \setlist[itemize,1]{label=\textbullet}

\newenvironment{mydesc}[1]
  {\list{}{\renewcommand\makelabel[1]{##1\hfil}\setlength{\itemsep}{1pt}%
     \settowidth\labelwidth{\makelabel{#1}}%
     \setlength\leftmargin{\dimexpr\labelwidth+\labelsep\relax}}}
  {\endlist}


\usepackage{hhline}
\usepackage{adjustbox}


\makeatletter
\def\lst@MSkipToFirst{%
    \global\advance\lst@lineno\@ne
    \ifnum \lst@lineno=\lst@firstline
        \def\lst@next{\lst@LeaveMode \global\lst@newlines\z@
        \lst@OnceAtEOL \global\let\lst@OnceAtEOL\@empty
        \lst@InitLstNumber % Added to work with modified \lsthk@PreInit.
        \lsthk@InitVarsBOL
        \c@lstnumber=\numexpr-1+\lst@lineno % this enforces the displayed line numbers to always be the input line numbers
        \lst@BOLGobble}%
        \expandafter\lst@next
    \fi}
\makeatother

\usepackage{tikz-timing}
\usepackage{graphbox}% http://ctan.org/pkg/adjustbox
% \usepackage[10pt]{rail}
% \input{/home/faculty/lpyeatt/courses/VHDL_book/newrails}



\definecolor{DarkBlue}{rgb}{0.0, 0.0, 0.55}


\title{Temporal PIG: Improving PIG to Resist the Noisy-TV Problem}
\subtitle{Thesis Defense}
\author{David Mathews}
\institute[SDSMT]{South Dakota School of Mines and Technology}

\bibliographystyle{IEEEtran}


%\usetheme{Warsaw}
\usepackage{bm}
\usepackage{amsmath}
%\usepackage{newtxtext,newtxmath}
\usepackage{array}
\usepackage{subfigure}
\setlength{\extrarowheight}{2pt}

\newcommand{\real}{\mathbb{R}}
\newcommand{\norm}[1]{\left\lVert#1\right\rVert}

\usepackage{calligra}
\usepackage{hyperref}

\DeclareMathOperator*{\argmax}{arg\,max}

\begin{document}
	
	\begin{frame}{}
		\maketitle
	\end{frame}
	
	%------------------------------------------------------------------
	
	\begin{frame}[fragile]{Outline}
		\begin{itemize}
			\item {Reinforcement Learning}
			\item {Intrinsic Motivation}
			\item {Noisy-TV}
			\item {Predicted Information Gain(PIG)}
			\item {Temporal PIG (TPIG)}
			\item {Methods}
			\item {Results}
			\item {Conclusions}
		\end{itemize}
	\end{frame}
	
	%------------------------------------------------------------------
	
	\begin{frame}[fragile]{Reinforcement Learning}
		\begin{itemize}
			\item {Agent}
			\item {Environment}
			\item {$a \in A$ - Actions}
			\item {$s \in S$ - States}
			\item { $\Theta_{s,a,s'}$ - Transition probability from $s$ to $s'$ after taking $a$.}
		\end{itemize}
	\end{frame}
	
	%------------------------------------------------------------------
	
	\begin{frame}[fragile]{Intrinsic Motivation}
		\begin{itemize}
			\item {Sparse environmental rewards make learning hard}
			\begin{itemize}
				\item {Robot Exploring a Maze}
			\end{itemize}
			\item {Internal reward, not external reward}
			\begin{itemize}
				\item {Reward agent for exploring!}
				\item {Prone to noise}
			\end{itemize}
		\end{itemize}
	\end{frame}
	
	%------------------------------------------------------------------
	
	\begin{frame}[fragile]{Noisy-TV}
		\begin{itemize}
			\item {State that creates noise when specific action is taken}
			\item {State has Noisy-TV and remote control}
			\begin{itemize}
				\item {When control is pressed, flips to random channel on TV}
				\item {High novelty, low value}
			\end{itemize}
			\item {Noisy-TV distracts most Intrinsic Reward systems}
		\end{itemize}
	\end{frame}
	
	%------------------------------------------------------------------
	
	\begin{frame}[fragile]{Kullback-Leibler Divergence}
		\[D_{KL} (P || Q) := \sum_{x \in X} P(x) \log(\frac{P(x)}{Q(x)})\]
		
		\begin{itemize}
			\item {KL-divergence calculates how similar two probability distributions are}
			\item {Value is large if not similar and close to 0 if similar}
			\item {Commonly used to measure information gain}
			\begin{itemize}
				\item {Information Gain from using P instead of Q}
			\end{itemize}
		\end{itemize}
	\end{frame}
	%------------------------------------------------------------------
	
	\begin{frame}[fragile]{Predicted Information Gain (PIG)}
		\[D_{KL} (\Theta_{as\cdot} || \hat{\Theta}_{as\cdot}) := \sum_{s' = 1}^{ N} \Theta_{ass'} \log_{2}(\frac{\Theta_{ass'}}{\hat{\Theta}_{ass'}})   \]
		
		\[ PIG(a,s) := \sum_{s*} \hat{\Theta}_{ass^{*}} D_{KL}(\hat{\Theta}_{as\cdot}^{a,s \rightarrow s^{*}} || \hat{\Theta}_{as\cdot}) \]
		\begin{itemize}
			\item {Exploration Algorithm using Information Theory}
			\item {Uses KL-divergence to calculate difference in Probability Distributions}
			\item {Takes the action that gives the agent the most information}
			\item {Still suffers from Noisy-TV}
		\end{itemize}
	\end{frame}
	%------------------------------------------------------------------
	
	\begin{frame}[fragile]{Temporal Predicted Information Gain (TPIG)}
		\[Reward(s,a,s') := D_{KL}(\hat{\Theta}_{a,s \rightarrow s'}^{t-1} || \hat{\Theta}_{as\cdot}^{t-1}) - D_{KL}(\hat{\Theta}_{a,s \rightarrow s'}^{t} || \hat{\Theta}_{as\cdot}^{t})\]
		\begin{itemize}
			\item {Goal: Make PIG less prone to Noisy-TV}
			\item {How? Confirmation.}
			\begin{itemize}
				\item {After each time-step, recalculate PIG using new model}
				\item {Compare calculations before and after time step}
				\item {If information was learned, new calculation should be less than old one, since less information is left}
				\item {Use this difference as Intrinsic reward instead}
			\end{itemize}
		\end{itemize}
	\end{frame}
	
	%------------------------------------------------------------------
	\begin{frame}[fragile]{Temporal Predicted Information Gain (TPIG)}
		\[Q(s,a) \leftarrow Q(s,a) + \alpha [r + \gamma *  \argmax_{a'} Q(s',a') + Q(s,a)]\]
		\[Q_{new} = Q_{old} + \alpha(Reward + \gamma Q_{new} - Q_{old})\]
		
		\[ TPIG(s) = \argmax_a  Q(s,a)\]
		\begin{itemize}
			\item Create model using temporal difference learning
			\begin{itemize}
				\item {$\alpha$ is learning rate (0.9)}
				\item {$\gamma$ is discount factor (0.9)}
			\end{itemize}
			\item Update the model using internal rewards.
		\end{itemize}
	\end{frame}
	
		
	%------------------------------------------------------------------
	
	\begin{frame}[fragile]{Epsilon Greedy PIG and TPIG}
		\begin{itemize}
			\item {$\epsilon$-greedy algorithms have an $\epsilon$\% chance of taking random action instead}
			\item {Helps algorithms avoid loops and explore new areas.}
			\item {$\epsilon$ is 10\% by default}
			\item {EPIG and ETPIG}
		\end{itemize}
	\end{frame}
	
	%------------------------------------------------------------------
	\begin{frame}[fragile]{Methods}
		\begin{itemize}
			\item {Test PIG, TPIG, EPIG, ETPIG, and Random Action Baseline}
			\item {Comparison Metrics}
			\begin{itemize}
				\item {Learning Rate (Amount explored in environment)}
				\item {Distraction Rate (How many times Noisy TV is used)}
				\item {Internal Model Accuracy (KL-divergence between True Model and Agent's Model)}
			\end{itemize}
		\end{itemize}
	\end{frame}
	
	%------------------------------------------------------------------
	
	\begin{frame}[fragile]{Simulation Details}
		\begin{itemize}
			\item {Gridworld Simulation}
			%\item {Random Starting state $s$ from set of starting states $S$}
			\item {Every state has 4 actions (up, down, left, right)}
			\item {Moving into wall will return agent to its current state}
			\item {Moving into Noisy-TV will set Noisy-TV state variable to random number}
		\end{itemize}
	\end{frame}
	
	%------------------------------------------------------------------
	\begin{frame}[fragile]{Simulations}
		\begin{figure}
			\begin{center}
				\includegraphics[scale=0.70]{"../images/4-TV.pdf"}
				\includegraphics[scale=0.70]{"../images/1-TV.pdf"}
				\includegraphics[scale=0.70]{"../images/No-TV.pdf"}
			\end{center}
			\caption{Three example simulations. Each semi-circle is a Noisy-TV. Arrows are one way walls.}
			\label{Fig:Sim}
		\end{figure}
	\end{frame}
	
	%------------------------------------------------------------------
	\begin{frame}[fragile]{Results (Distraction Rate)}
		\begin{figure}
			\begin{center}
				\includegraphics[scale=0.40]{"../images/Distraction_Rate_4-TV.pdf"}
				\includegraphics[scale=0.40]{"../images/Distraction_Rate_1-TV.pdf"}
			\end{center}
			\caption{Distraction Rate of all Five Policies}
		\end{figure}
	\end{frame}
	
	%------------------------------------------------------------------
	
	\begin{frame}[fragile]{Results (Distraction Rate)}
		\begin{figure}
			\begin{center}
				\includegraphics[scale=0.40]{"../images/Distraction_Rate_No-TV.pdf"}
				\includegraphics[scale=0.40]{"../images/Distraction_Rate_1-TV.pdf"}
			\end{center}
			\caption{Distraction Rate of all Five Policies}
		\end{figure}
	\end{frame}
	
	%------------------------------------------------------------------
	
	\begin{frame}[fragile]{Results (Learning Rate)}
		\begin{figure}
			\begin{center}
				\includegraphics[scale=0.40]{"../images/Missed_States_4-TV.pdf"}
				\includegraphics[scale=0.40]{"../images/Missed_States_1-TV.pdf"}
			\end{center}
			\caption{Learning Rate of all Five Policies}
		\end{figure}
	\end{frame}
	
	%------------------------------------------------------------------
	
	\begin{frame}[fragile]{Results (Learning Rate)}
		\begin{figure}
			\begin{center}
				\includegraphics[scale=0.40]{"../images/Missed_States_No-TV.pdf"}
				\includegraphics[scale=0.40]{"../images/Missed_States_1-TV.pdf"}
			\end{center}
			\caption{Learning Rate of all Five Policies}
		\end{figure}
	\end{frame}
	
	%------------------------------------------------------------------
	
	
	\begin{frame}[fragile]{Results (Model Accuracy 4-TV)}
		\begin{figure}
			\begin{center}
				\includegraphics[scale=0.40]{"../images/Missed_States_4-TV.pdf"}
				\includegraphics[scale=0.40]{"../images/Model_Accuracy_4-TV.pdf"}
			\end{center}
		\end{figure}
	\end{frame}
	
	%------------------------------------------------------------------
	
	\begin{frame}[fragile]{Results (Model Accuracy 1-TV)}
		\begin{figure}
			\begin{center}
				\includegraphics[scale=0.40]{"../images/Missed_States_1-TV.pdf"}
				\includegraphics[scale=0.40]{"../images/Model_Accuracy_1-TV.pdf"}
			\end{center}
			\caption{Model Accuracy of all Five Policies}
		\end{figure}
	\end{frame}
	
	%------------------------------------------------------------------
	
	\begin{frame}[fragile]{Results (Model Accuracy No-TV)}
		\begin{figure}
			\begin{center}
				\includegraphics[scale=0.40]{"../images/Missed_States_No-TV.pdf"}
				\includegraphics[scale=0.40]{"../images/Model_Accuracy_No-TV.pdf"}
			\end{center}
			\caption{Model Accuracy of all Five Policies}
		\end{figure}
	\end{frame}
	
	%------------------------------------------------------------------
	
	\begin{frame}[fragile]{Results (Epsilon Comparison 4-TV)}
		\begin{figure}
			\begin{center}
				\subfigure[4-TV EPIG Simulation \label{Fig:DRECEP4TV}]{\includegraphics[scale=0.4]{"../images/Epsilon_Distractions_EPIG_4-TV.pdf"}}			
				\subfigure[4-TV ETPIG Simulation \label{Fig:DRECET4TV}]{\includegraphics[scale=0.4]{"../images/Epsilon_Distractions_ETPIG_4-TV.pdf"}}
			\end{center}
			\caption{Distraction Rate Epsilon Comparison EPIG VS ETPIG 4-TV Simulation}
			\label{Fig:DREC4TV}
		\end{figure}
	\end{frame}
	
	%------------------------------------------------------------------
	
	\begin{frame}[fragile]{Results (Epsilon Comparison 1-TV)}
		\begin{figure}	
			\begin{center}
				\subfigure[1-TV EPIG Simulation \label{Fig:DRECEP1TV}]{\includegraphics[scale=0.4]{"../images/Epsilon_Distractions_EPIG_1-TV.pdf"}}
				\subfigure[1-TV ETPIG Simulation \label{Fig:DRECET1TV}]{\includegraphics[scale=0.4]{"../images/Epsilon_Distractions_ETPIG_1-TV.pdf"}}
			\end{center}
			\caption{Distraction Rate Epsilon Comparison EPIG VS ETPIG 1-TV Simulation}
			\label{Fig:DREC1TV}
		\end{figure}
	\end{frame}
	
	%------------------------------------------------------------------
	
	\begin{frame}[fragile]{Results (Epsilon Comparison No-TV)}
		\begin{figure}	
			\begin{center}
				\subfigure[No-TV EPIG Simulation \label{Fig:DRECEP0TV}]{\includegraphics[scale=0.4]{"../images/Epsilon_Distractions_EPIG_No-TV.pdf"}}				
				\subfigure[No-TV ETPIG Simulation \label{Fig:DRECET0TV}]{\includegraphics[scale=0.4]{"../images/Epsilon_Distractions_ETPIG_No-TV.pdf"}}
			\end{center}	
			\caption{Distraction Rate Epsilon Comparison EPIG VS ETPIG No-TV Simulation}
			\label{Fig:DREC0TV}
		\end{figure}
	\end{frame}
	
	%------------------------------------------------------------------
	
	\begin{frame}[fragile]{Results (Epsilon Comparison 4-TV)}
		\begin{figure}
			\begin{center}
				\subfigure[4-TV EPIG Simulation \label{Fig:EMECEP4TV}]{\includegraphics[scale=0.4]{"../images/Epsilon_Missed_States_EPIG_4-TV.pdf"}}
				\subfigure[4-TV ETPIG Simulation \label{Fig:EMECET4TV}]{\includegraphics[scale=0.4]{"../images/Epsilon_Missed_States_ETPIG_4-TV.pdf"}}
			\end{center}
			\caption{Missed States Epsilon Comparison EPIG VS ETPIG 4-TV Simulation}
			\label{Fig:EMEC4TV}
		\end{figure}
	\end{frame}
	
	%------------------------------------------------------------------
	
	\begin{frame}[fragile]{Results (Epsilon Comparison 1-TV)}
		\begin{figure}
			\begin{center}
				\subfigure[1-TV EPIG Simulation \label{Fig:EMECEP1TV}]{\includegraphics[scale=0.4]{"../images/Epsilon_Missed_States_EPIG_1-TV.pdf"}}
				\subfigure[1-TV ETPIG Simulation \label{Fig:EMECET1TV}]{\includegraphics[scale=0.4]{"../images/Epsilon_Missed_States_ETPIG_1-TV.pdf"}}
			\end{center}
			\caption{Missed States Epsilon Comparison EPIG VS ETPIG 1-TV Simulation}
			\label{Fig:EMEC1TV}
		\end{figure}
	\end{frame}
	
	%------------------------------------------------------------------
	
	\begin{frame}[fragile]{Results (Epsilon Comparison No-TV)}
	\begin{figure}
		\begin{center}
			\subfigure[No-TV EPIG Simulation \label{Fig:EMECEP0TV}]{\includegraphics[scale=0.4]{"../images/Epsilon_Missed_States_EPIG_No-TV.pdf"}}
			\subfigure[No-TV ETPIG Simulation \label{Fig:EMECET0TV}]{\includegraphics[scale=0.4]{"../images/Epsilon_Missed_States_ETPIG_No-TV.pdf"}}
		\end{center}
		\caption{Missed States Epsilon Comparison EPIG VS ETPIG No-TV Simulation}
		\label{Fig:EMEC0TV}
	\end{figure}
	\end{frame}
	
	%------------------------------------------------------------------
	
	\begin{frame}[fragile]{Results (Epsilon Comparison 4-TV)}
		\begin{figure}
			\begin{center}
				\subfigure[4-TV EPIG Simulation \label{Fig:EAECEP4TV}]{\includegraphics[scale=0.4]{"../images/Epsilon_Model_Accuracy_EPIG_4-TV.pdf"}}\hfill
				\subfigure[4-TV ETPIG Simulation \label{Fig:EAECET4TV}]{\includegraphics[scale=0.4]{"../images/Epsilon_Model_Accuracy_ETPIG_4-TV.pdf"}}\hfill
			\end{center}
			\caption{Model Accuracy Epsilon Comparison EPIG VS ETPIG 4-TV Simulation}
			\label{Fig:EAEC4TV}
		\end{figure}
	\end{frame}
	
	%------------------------------------------------------------------
	
	\begin{frame}[fragile]{Results (Epsilon Comparison 1-TV)}
		\begin{figure}
			\begin{center}
				\subfigure[1-TV EPIG Simulation \label{Fig:EAECEP1TV}]{\includegraphics[scale=0.4]{"../images/Epsilon_Model_Accuracy_EPIG_1-TV.pdf"}}\hfill
				\subfigure[1-TV ETPIG Simulation \label{Fig:EAECET1TV}]{\includegraphics[scale=0.4]{"../images/Epsilon_Model_Accuracy_ETPIG_1-TV.pdf"}}\hfill
			\end{center}
			\caption{Model Accuracy Epsilon Comparison EPIG VS ETPIG 1-TV Simulation}
			\label{Fig:EAEC1TV}
		\end{figure}
	\end{frame}
	
	%------------------------------------------------------------------
	
	\begin{frame}[fragile]{Results (Epsilon Comparison No-TV)}
		\begin{figure}
			\begin{center}
				\subfigure[No-TV EPIG Simulation \label{Fig:EAECEP0TV}]{\includegraphics[scale=0.4]{"../images/Epsilon_Model_Accuracy_EPIG_No-TV.pdf"}}\hfill
				\subfigure[No-TV ETPIG Simulation \label{Fig:EAECET0TV}]{\includegraphics[scale=0.4]{"../images/Epsilon_Model_Accuracy_ETPIG_No-TV.pdf"}}\hfill
			\end{center}
			\caption{Model Accuracy Epsilon Comparison EPIG VS ETPIG No-TV Simulation}
			\label{Fig:EAEC0TV}
		\end{figure}
	\end{frame}
	
	%------------------------------------------------------------------
	\begin{frame}[fragile]{Conclusions}
		\begin{itemize}
			\item {TPIG performs better than PIG for all three testing metrics}
			\begin{itemize}
				\item {Distraction Rate}
				\item {Learning Rate}
				\item {Model Accuracy}
			\end{itemize}
			\item {PIG learns faster initially, but TPIG creates a better model for long simulations}
			\item {Epsilon can further decrease the Distraction Rate, but at the cost of model accuracy}
		\end{itemize}
	\end{frame}
	
	%------------------------------------------------------------------
	\begin{frame}[fragile]{Future Work}
		\begin{itemize}
			\item {TPIG can be expanded to continuous domain}
			\item {Compare TPIG to other Reinforcement Learning algorithms}
			\item {Expand TPIG to also use external reward}
		\end{itemize}
	\end{frame}
	
	%------------------------------------------------------------------
	
	\begin{frame}[fragile]{Questions?}
	\end{frame}
	
	%------------------------------------------------------------------
	
	\end{document}
	\input{..\images}
							
